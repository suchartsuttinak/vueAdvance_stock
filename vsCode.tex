{
    "files.associations": { "*.module": "php" },
    "intelephense.diagnostics.undefinedTypes": false,

    "[javascript]": {
        "editor.formatOnSave": true,
        "editor.formatOnPaste": false,
        "editor.formatOnType": false,
        "editor.defaultFormatter": "esbenp.prettier-vscode"
        },
        "terminal.integrated.defaultProfile.windows": "Command Prompt",
        "workbench.iconTheme": "material-icon-theme",
        
        
            "files.exclude": {
                "**/.git": false,
            },
            "tailwindCSS.emmetCompletions": true,
            "tailwindCSS.includeLanguages": {
            "plaintext": "javascript",
            "javascript": "javascript",
            "html": "html"
    },
            "window.zoomLevel": 1, 
          
            "phpfmt.php_bin": "\"C:\\xampp\\php\\php.exe\"",
            "[php]": {
                "editor.formatOnSave": false,
                "editor.defaultFormatter": "kokororin.vscode-phpfmt"
            },
            "phpfmt.passes": [
                "AlignGroupDoubleArrow",
                "AllmanStyleBraces",
                "AlignEquals",
                "IndentTernaryConditions",
                "MergeElseIf",
                "NewLineBeforeReturn",
                "RemoveSemicolonAfterCurly",
                "SmartLnAfterCurlyOpen"
            ],

            "vetur.validation.template": false,
            "vetur.validation.script": false,
            "vetur.validation.style": false,
            
            -----------------
            =========================
git pull origin maste
1. ติดตั้ง git
 Check status = git version

2. ก าหนดข้อมูลผู้ใช้
git config --global user.name "suchart"
git config --global user.email suchartsuttinak@gmail.com

3. คำสั่งเช็คข้อมูลที่ก าหนดไว้
git config --list
git config --global --list

การสรัาง git ลงในโปรเจ็ค
4.git init

5.Open Project in VS Code
   = code .

6. การตั้งค่าให้มองเห็นโฟล์เดอร์ git 
 ไปที่ Seting -> seting json พิมพ์โค๊ด
{
"files.exclude": {
    "**/.git": false,
},

7. เรียกดูสถานะไฟล์ ว่าอยู่ในสภานะใด
 git status

8. ทำการ trackfile เข้าสู่ staging area 
 git add .

9. การลบออกจาก staging area
---
git reset greeting.txt
git reset folder_name กรณีอยุ่ในโฟล์เดอร์
(git reset src\coding.txt)

---------------------------------------------------------------
10. การบันทึกประวัติการทำงาน (code)
---
git commit -m "first commit" (ต้องใส่คอมเม้นท์เสมอ)

11. เรียกประวัติที่บันทึกไว้
---
git log


12. การย้อนประวัติด้วย git checkout (ประวัติเดิมจะไม่หาย)
---
git checkout 0de0b91


13. การเรียกดูประวัติที่ถูกย้อนข้ามไปแล้ว
---
git reflog


14. หากต้องการย้อนประวัติแบบไม่เก็บของเดิมไว้  (คล้ายกับ git checkout) แต่ทำให้ไม่เปลืองพื้นที่ git
---
git reset --hard 0de0b91


15.หากต้องการย้อนประวัติแบบเก็บของเดิมไว้
git reset --soft 0de0b91

--------------------------------------------------------------
วันที่ 2

1. เปิดโปรเจ็ค set up git

2.git init

Step 3: เปลี่ยนชื่อ branch จาก master เป็น main
---
git branch -M main

Step 4: คำสั่งเรียกดู branch ทั้งหมด
---
git branch

Step 5: การสร้าง branch ใหม่
---
git branch dev


Step 6: เปลี่ยนไปทำงานที่ branch dev
---
git checkout dev

เราลอง checkout กลับไปที่ main
---
git checkout main


Step 7: ลองสร้าง branch ใหม่ แล้วทำการ checkout ไปที่ branch ที่สร้างทันที
---
git checkout -b feature


Step 8: การรวม branch  ย่อย กลับเข้า branch หลัก
---
git merge feature --no-ff (เก็บข้อมูลเก่าไว้)

git merge feature (ไม่เก็บข้อมูลเดิมไว้)

Step 9: รวม branch เข้า main
---
git merge dev 
--------------------------------------------------------------

Steip 10: สร้างไฟล์ .gitignore
---
.gitignore

วีธีแก้ปัญหาเมื่อเพิ่ม  .gitignore  ภายหลัง
กรณีสร้างไฟล์ .gitignore ภายหลัง เราจะต้องแก้ปํญหาด้วยคำสั่งดังนี้
---
ก่อนใช้คำสั่งด้านล่างควร commit ให้เรียบร้อยก่อน
git rm -rf --cached . (เป็นคำสั่งที่ cached และเพิ่มข้อมูลเข้าไปใหม่)
git add .
--------------------------------------------------------------
 
Step 11: add remote (เข้าไปที่  Web:gitHub)
---
ที่ comman 
git remote add origin https://github.com/iamsamitdev/git-branch.git

หากต้องการเช็คว่า ตัวโปรเจ็กต์เราผูกอยู่ที่ remote ไหน
---
git remote show origin (ต้องจำคำสั่งนี้ให้ได้)

Step 12: การอัพโหลดโค๊ดขึ้น remote (github) server
---
git push -u origin main

การดึงข้อมูลจาก เซิฟเวอร์มาที่เครื่อง
git pull origin master
-------------------------------------------------------

วิธีการนำ Code ไปใช้ที่อื่น
1. นำไฟล์จาก gitHub เลือกโปรเจ็คที่ต้องการ คัดลอกลิ้งค์ https://github.com/suchartsuttinak/git-workshop.git
2.https://github.com/suchartsuttinak/git-workshop.git
3.เลือกตำแหน่งที่จะวางโค๊ด  คลิ๊กขวา แล้วเลือก git Bash Hear
4. พิมพ์คำสั่งลงไป = git clone https://github.com/suchartsuttinak/git-workshop.git


วันที่ 1
=========================

การใช้งาน Git Step by Step
-------------------------------------------
1. การสร้าง git ลงในโปรเจ็กต์
---
git init

2. เปิดโปรเจ็กต์เข้า VS Code
---
code .

3. สร้างไฟล์ต่างๆ ในโปรเจ็กต์ของเรา
---
.git
src
   |-- coding.txt
greeting.txt

4. เรียกดูสถานะไฟล์
---
git status

5. ทำการ trackfile เข้าสู่ staging area
---
git add greeting.txt
git add .
git add src/*.txt

6. การลบออกจาก staging area
---
git reset greeting.txt
git reset folder_name

7. การบันทึกประวัติการทำงาน (code)
---
git commit -m "first commit"

8. เรียกประวัติที่บันทึกไว้
---
git log

9. หากต้องการลบ commit ออก
---
git update-ref -d HEAD

10. การย้อนประวัติด้วย git checkout
---
git checkout 0de0b91

11. การเรียกดูประวัติที่ถูกย้อนข้ามไปแล้ว
---
git reflog

12. หากต้องการย้อนประวัติแบบไม่เก็บของเดิมไว้
---
git reset --hard 0de0b91

=========================
วันที่ 2
=========================
การทำงานกับ Git Branch

Step 0: เปิดโปรเจ็กต์ที่ต้องการเข้ามาใน VS Code
---
code .

Step 1: Setup git ลงในโปรเจ็กต์
---
git init

Step 2: Track file ในโปรเจ็กต์
---
git add .

Step 3: Commit file ในโปรเจ็กต์
---
git commit -m "first commit"

Step 4: เปลี่ยนชื่อ branch จาก master เป็น main
---
git branch -M main

Step 5: คำสั่งเรียกดู branch ทั้งหมด
---
git branch

Step 6: การสร้าง branch ใหม่
---
git branch dev

Step 7: เปลี่ยนไปทำงานที่ branch dev
---
git checkout dev

เราลอง checkout กลับไปที่ main
---
git checkout main

Step 8: ลองสร้างใหม่ แล้วทำการ checkout ไปที่ branch ที่สร้างทันที
---
git checkout -b feature


Step 9: การรวม branch  ย่อย กลับเข้า branch หลัก
---
git merge feature --no-ff

Step 10: รวม branch เข้า main
---
git merge dev

Steip 11: สร้างไฟล์ .gitignore
---
.gitignore

กรณีสร้างไฟล์ .gitignore ภายหลัง เราจะต้องแก้ปํญหาด้วยคำสั่งดังนี้
---
ก่อนใช้คำสั่งด้านล่างควร commit ให้เรียบร้อยก่อน
git rm -rf --cached .
git add .

Step 12: add remote
---
git remote add origin https://github.com/iamsamitdev/git-branch.git

หากต้องการเช็คว่า ตัวโปรเจ็กต์เราผูกอยู่ที่ remote ไหน
---
git remote show origin

Step 13: การอัพโหลดโค๊ดขึ้น remote (github) server
---
git push -u origin main














       
            
}
